\documentclass[11pt]{article}
\usepackage{graphicx}
\usepackage{amssymb}
\usepackage{amsmath}
\usepackage{epstopdf}
\usepackage{epsfig}
\usepackage[frenchb]{babel}
\usepackage[utf8]{inputenc}
\usepackage{graphicx}
\DeclareGraphicsRule{.tif}{png}{.png}{`convert #1 `basename #1 .tif`.png}
\usepackage{bm}
\usepackage{mathabx}

%\textwidth = 6.5 in
%\textheight = 9 in
%\oddsidemargin = 0.0 in
%\evensidemargin = 0.0 in
%\topmargin = 0.2 in
%\headheight = 0.1 in
%\parindent = 0.2 in



\title{Weighted LISA (Local indicator of spatial autocorrelation): sketch, v1}
%
% \toctitle specifies the title as will be printed in the table of 
% contents. Use \protect\newline to force a line break in your title.
\author{F.Bavaud, UNIL}

\date{July 2020}
 


\begin{document}
\maketitle   

\section{Introduction}
To perform local (or global) weighted auto-correlation, two ingredients are needed: 
\begin{itemize}
  \item[$\bullet$] a $n\times n$  exchange matrix $\bm{E}=(e_{ij})$ between ``positions" $i$ and $j$, which is symmetric, non-negative, and normalized to $e_{\bullet\bullet}=1$. In addition,  $\bm{E}$ has to be {\em weight-compatible} in the sense 
  that the weights $\bm{f}=(f_i)$ defined by $f_i:=e_{i\bullet}>0$ are the relevant {\em strictly positive} weights under consideration. 
  \item[$\bullet$] a $n\times n$  matrix of squared Euclidean distances $\bm{D}=(d_{ij})$. 
\end{itemize}

 \subsection{Markov chains}
 Let $\bm{\Pi}:=\mbox{diag}(\bm{f})$. Define $\bm{W}:=c^{-1}\bm{E}$, that is $w_{ij}:=\frac{e_{ij}}{f_i}$. By construction, $\bm{W}$ is the $n\times n$ transition matrix of a reversible Markov chain (we assume to be {\em regular}, that is irreducible and aperiodic, with stationary distribution $\bm{f}$. It obeys $\bm{\Pi}\bm{W}=\bm{W}^\top\bm{\Pi}=\bm{E}$.
 
 
 
 \subsection{Relative autocorrelation index $\delta$}
 Define the global and local inertia by 
\begin{equation}
\label{ }
 \Delta:=\frac12\sum_{ij}f_if_jd_{ij}
 \qquad\qquad
  \Delta_{\mbox{\tiny loc}}:=\frac12\sum_{ij}e_{ij}d_{ij}
\end{equation} 
The {\em relative autocorrelation index $\delta$} (a weighted, multivariate generalization of  Moran's $I$) is 
\begin{equation}
\label{ }
\delta:=\frac{\Delta- \Delta_{\mbox{\tiny loc}}}{\Delta}\in [-1,1]
\end{equation}
 
 \subsection{Local autocorrelation index $\delta_i$ (LISA)}
 There are many ways to define a local autocorrelation index $\delta_i$ such that $\delta=\sum_{i=1}^n f_i \delta_i$.  Presumably the most elegant (unpublished, but cited in ``Flow autocorrelation: a dyadic approach" by 
F. Bavaud, M. Kordi, C. Kaiser, The Annals of Regional Science (2018) Vol. 61, Issue 1, pp 95--111, https://doi.org/10.1007/s00168-018-0860-y) is 
\begin{equation}
\label{ }
\delta_i:=\frac{(\bm{W}\bm{B})_{ii}}{\Delta}
\end{equation}
where $\bm{B}:=-\frac12 \bm{H}\bm{D}\bm{H}^\top$ is the matrix of (unweighted) {\em scalar products} and $\bm{H}:=\bm{I}_n-\bm{1}_n\bm{f}^\top$ is the (idempotent, generally non symmetric) weighted {\em centration matrix}. 

\

By construction, $\bm{H}\bm{1}_n=\bm{0}_n$, and $\bm{H}^\top\bm{f}=\bm{0}_n$.

\

Note that $\bm{E}\bm{H}=\bm{E}-\bm{f}\bm{f}^\top$ and $\bm{H}^T\bm{E}=\bm{E}-\bm{f}\bm{f}^\top$ and finally \\ $\bm{H}^T\bm{E}\bm{H}=\bm{E}-\bm{f}\bm{f}^\top$ 

\

By construction, 
\begin{align*} 
\sum_i f_i (\bm{W}\bm{B})_{ii} & = \mbox{trace}(\bm{\Pi}\bm{W}\bm{B})=\mbox{trace}(\bm{E}\bm{B})=-\frac12\mbox{trace}(\bm{E}\bm{H}\bm{D}\bm{H}^\top) \\ 
  & =-\frac12\mbox{trace}(\bm{H}^\top\bm{E}\bm{H}\bm{D})=\frac12\mbox{trace}((\bm{f}\bm{f}^\top-\bm{E})\bm{D})=\Delta- \Delta_{\mbox{\tiny loc}}
\end{align*}
which proves  $\delta=\sum_{i=1}^n f_i \delta_i$. 
 
 
 
 \end{document}